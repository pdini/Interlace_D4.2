\chapter{Introduction}
\label{ch:Introduction}

\vspace{-1cm}
\begin{center}
Giuseppe Littera and Paolo Dini
\end{center}

The component testing of the platform was performed only at an essential level, as reported in deliverable D3.2 \cite{INTERLACE_D32}, due to lack of time and funding. Not having had the time to perform the component acceptance testing, we could not perform the field tests either. We expect the component testing to be completed sometime in 2019 and the field testing probably in 2020. All updates will be shared on GitHub\footnote{\url{https://github.com/InterlaceProject/InterlaceBlockchain}} and/or on the project's website.\footnote{\url{https://www.interlaceproject.eu/}} Here we add a few words on the commercialisation plan, which has evolved significantly over the past few years.

The overarching objective of the project was to develop, collaboratively, an open source and blockchain-based transactional platform for mutual credit. This was motivated by a desire to build in-house blockchain expertise in order to remain competitive, but also in order to achieve better scalability as the Sardex platform is extended to additional circuits in Italy and beyond. Although INTERLACE has laid the groundwork for such a system, as shown in deliverable D2.3 \cite{INTERLACE_D23} the system involves many other components beyond the transactional platform. Therefore, its overhaul and redesign is going to take a significantly greater amount of resources than this project could provide. It follows that the future business model and commercialisation plan need to be convincing for the size of venture capital (VC) investment needed. The next chapter provides a brief outline of a possible plan.

\newpage
